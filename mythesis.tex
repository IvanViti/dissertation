\documentclass[12pt]{niuthesis}
% \documentclass[12pt,singlespacing]{niuthesis}

% some packages are loaded here
\usepackage{latexsym}		% to get LASY symbols
\usepackage{graphicx}		% to insert PostScript figures
\usepackage{rotating}           % defines sidways table and figure env.
% \usepackage{hyperref}		% to insert hyperrefs
\usepackage{graphicx}
\usepackage{varwidth}
\usepackage{algorithm}
\usepackage{algpseudocode}

%%%%%%%%%%%%%%%%%%%%%%%%%%%%%%%%%%%%%%%%%%%%%%%%%%%%%%%%%%%%%%%%
% Some LaTeX macros

\newcommand{\twochoices}[2]{\left\{ \begin{array}{lcc}
        \displaystyle #1 \\ \vspace{-10pt} \\
        \displaystyle #2 \end{array} \right. } %}

\newcommand{\twovec}[2]{\left(\begin{array}{c} #1 \\ #2 \end{array}\right)}

\newcommand{\twomatrix}[4]{\left(\begin{array}{cc} #1 & #2 \\ 
     #3 & #4 \end{array}\right)}

% Uncomment the following line for a List of Symbols
% \newcommand{\listofXXX}{\input{symbols}}

%%%%%%%%%%%%%%%%%%%%%%%%%%%%%%%%%%%%%%%%%%%%%%%%%%%%%%%%%%%%%%%%%%
% Choose the chapter(s) / files you want to work with

%%% use this to compile all chapters
\def\files{Introduction,Theory,TheoryVortex,Dias,GLcode,refs}

%%% use this to work with only one chapter
% \def\files{ch2}

\includeonly{\files}

%%%%%%%%%%%%%%%%%%%%%%%%%%%%%%%%%%%%%%%%%%%%%%%%%%%%%%%%%%%%%%%%%
\begin{document}

\title{GPU simulations in Multiphasic Nanosolids and Superconducting Nanostructures}

\author{Ivan Viti}

\major{Physics}
\degree{Dissertation}{Ph.D.}{Doctor of Philosophy}
\degreedate{October}{2016}
\department{Department of Physics}
\director{Andreas Glatz}

\begin{abstract}
	With the ever-increasing availability of computing power, namely from Graphics Processing Units (GPUs), comes the responsibility to simulate more complicated systems. Complex functions such as the Ginzburg-Landau function can not be studied analytically for mesoscopic phenomena. Similarly, a thorough understanding of variable range hopping in electrons requires a Markov-Chain Monte-Carlo algorithm. With this in mind we computationally study two cases of condensed matter physics, thermoelectrics and superconductor-bound vortices.
Here, We develop and implement a novel algorithm for simulation of variable-range-hopping of electrons in a nanosolid. We also created an ad-hoc cluster in order to run these and other simulations. 
	Vortex-vortex interactions and vortex-inclusion interactions are not very well understood analytically. The mathematics becomes near-impossible when taken to mesoscopic scales. Applied temperature, magnetic field, or currents only serve to complicate the system. Yet these are all factors that need to be well understood before serious applications can take place. Here we report two basic systems, The effect of vortex-inclusion matching on the effective resistance, and a novel funnel system for slowing the vortices. We describe how matching the number of inclusions to the number of vortices can help reduce the amount of vortex-induced resistance. We also describe how an aperture-type system can help to slow down vortices as they are travelling through the system draining energy. 



\end{abstract}

\begin{acknowledgments}
  I would like to thank Andreas Glatz for his constant guidance. I would also like to thank the computer science department for allowing us to use their Gaea cluster for our simulations, as well as the Physics department for housing our Thoon cluster.
  
  This work is supported by a grant from the DoE.
\end{acknowledgments}

% comment this to suppress prologue
\MakeThesisPrologue % includes table of contents
% \tableofcontents

\chapter{Introduction}		% chapter 1
\label{introchap}

\label{section}
\section{Condensed Matter GPU simulations}
The parallel simulation of physical systems is the central pillar of this dissertation. From Clifford simulations, to cryptography, to Lennard-Jones gasses were studied during this time. This dissertation will focus on the two most promising systems, electron hop studies in thermoelectrics, and vortex jamming/pinning in superconductors. 

\section{Thermoelectrics}
Thermoelectric devices can be used to turn electric power into cooling or heating (Peltier effect). They can also be used to turn thermal gradients into electricity (Seebeck effect). Most modern refrigeration techniques require compression of some fluid which by defintion means moving parts. A Peltier device can do this much more simply. The way they work is similar to sweat on human skin. In a random assortment of water molecules on a surface, typically the "hot" ones will leave, thereby cooling that surface. If one adds a breeze, this process can occur much more efficiently. In a nutshell, this is what is happening in a Peltier device. Electrons are attempting to move heat from one place to another, but the whole process is artificially skeewed when a voltage is applied. Because this is an electron-based effect, we want materials with a high electric conductivity. On the other hand, if the thermal conductivity is too high, then the thermal gradient will suffer and heat engine efficiency will be diminished. Currently Peltier devices can only acheive 12\% of maximum theoretical efficiency compared to compressor refrigerators which can acheive 60\%. By constructing artificial nanosolids, we can manipulate which electrons can transfer heat, thereby dictating the thermal conductivity ~\cite{glatz09}. But for this to work, we need to figure out the correct specifications of these nanosolids. We could spend years building them and experimentally testing them, but with the underlying physics being well-understood it is easier to simulate them. For this, we wrote {\sc dias}. To differentiate ourselves from other Mott simulators, we did not limit ourselves to nearest neighbors.  By using Nvidia's {\sc cuda}, we could parallelize the code on GPUs which allowed all cell probabilities to be measured at the same time with no cost to performance. We used GTX-570 video cards to run these simulations. There are a lot of regimes to study in variable-range hopping.

\section{Vortex pinning}
Superconductors are valuable for an ever-expanding range of uses. Some prominent examples are magnets, qubits, voltage to frequency converters. The department of energy is considering the posibility of transfering large amounts of current losslessly across large distances. Theoretically this should be possible with superconductors. Practically we come across the problematic phenomena known as vortices.  In this dissertation we study this effect on the induced voltage in the system. A "hand-waving" approach to this helps us understand why vortices are important. They can be seen as a magnetic flux in the direction of the field. If a current is then applied perpendicular to this field, the vortices will move in a direction perpendicular to these two vectors. This movement will due to the Lorentz force, which will drain energy from the system. This drain is then measured from the voltage gap which appears in the material. Dr. Glatz wrote a program called {\sc GL} which models the important parameter function $\psi$, and displays all relevant observables. This program models a discretized system which is initialized with a parameter function and then each timestep is defined by following the Ginzburg-Landau equation. Using this program, we studied many situations such as grids of non-superconducting inclusions, funnels, vortex ratchets, and critical currents. 


		% file with Chapter 1 contents
\chapter{Variable Range Hopping}		% chapter 1
\label{theorychap}

\section{History}
In 1937, Nevill Mott and Rudolf Peierls explained why some materials which should have been conductors, acted as insulators. This was due to electron-electron interactions, which is not taken into account in band theory ~\cite{mott72}. From there, Miller and Abrahams proposed a network resistor model which Efros and Schklovskii built on ~\cite{efros75}. Because of the dynamic complexity of even a small system (a 10 by 10 system can have $2^{100}$ configurations) simulations are a large part of modern electron hopping studies ~\cite{kirkengen09}. The theory has gotten more consise over time. For the sake of simulation, much of it can be compressed into equation~\ref{probability}.

\section{Jump Probability}

In order to calculate were the electron will jump, we must first find the probability of each jump site. Our main equation involved is as follows:
\begin{eqnarray}
P_{ij} = \exp (-2\alpha R_{ij} -  E_{ij}/kT)
\label{probability}
\end{eqnarray}

where $k$ is the Boltzmann constant, $R_{ij}$ is the distance between cells, $E_{ij}$ is the energy change of the system if a particle were to move from i to j, and T is the temperature of the system. $E_{ij}$ has various inputs which vary in energy scale.
\begin{eqnarray}
E_{ij} =  \triangle U_{ij} e^2/a\kappa_1  + e f_i \triangle V_{ij} + f_i  \triangle \mu_{ij} + f_i (eV) \triangle x_{ij} 
\label{deltaE}
\end{eqnarray}
where $i$ refers to the starting site index and $j$ refers to the target site index,  $f$ is the occupation of a site, $a$ is the size of the granule, $\kappa_1$ is the intra-granule dielectric constant, $\mu$ is the substrate potential, $V_{ij} = \sum_{n=1}^{N} e^2/\kappa_2 r_{n}$ is the local potential from the rest of the particles, $\kappa_2$ is the inter-granule dielectric constant, $\triangle x$ is the component of r in the direction of eV , and $eV$ is an externally applied voltage. $\triangle U_{ij}$ is -1 if electrons stacked, 1 if electrons de-stacked, and 0 otherwise. The first two components of equation ~\ref{deltaE} together constitute the electrostatic component of this system. The most powerful is the Coulomb blockade. This introduces a large penalty into the probability of an electron occupying the same site as another electron. If the site at which an electron can travel to is empty, the chances of transport are higher than if the site is filled ~\cite{glazman05}. There is also a contribution from the substrate. There is an inherent randomness in the potential at different lattice sites and that is where this comes in. This energy landscape somewhat randomizes starting energies and can fill in as electron donor. There is a general electric potential which will try to get the electrons to space out evenly in the substrate (periodic boundary conditions). Finally there is a current portion which if a voltage bias is applied on the left and right sides of the system, then the probability of electrons hopping with that bias is increased ~\cite{aharony92}. The exponential term is artificially limited below 0 to keep the electron hop ranges realistic. Analytically, equation~\ref{probability} can be maximized to find the most common jump distance (take the derivative with respect to r and set equal to 0). In our case, there are 2 problems with this approach. First, we are more interested in the average jump distance which need not be the maximum. Second, The analytical approach assumes a continuous system, where in reality it is a discrete number of interacting points. There are a few nuances in the theory that need to be adressed.
\subsection{Mott vs. Effros-Schklovskii}
While the approach to describe physical systems by Mott vs Effros-Schklovskii (ES) theory are similar, there are a few key differences in the details. First, The localization parameters are different due to differences in the density of states. The localization is temperature dependent in the ES system. This is because ES considered the situation where if an electron tries to tunnel, It must leave an electron hole behind. This enhanced requirement for the energy means that at low temperatures, the density of states at the Fermi level vanishes ~\cite{joung}. The resistivity for the Mott system ends up being related to the temperature as

\begin{eqnarray}
\ln(\rho) \approx (T_o / T)^{1/4} ,
\label{fourth}
\end{eqnarray}
where $\rho$ is the resistivity and $T$ is the temperature. Meanwhile, in the ES system we have
\begin{eqnarray}
\ln(\rho) \approx (T_o / T)^{1/2}.
\label{half}
\end{eqnarray}
The Mott resistance comes from $\delta E = \alpha_1 / g r_{ij}^3 $, where $\alpha_i$ are prefactors of order unity. The ES resistance comes from the usual Coulomb blockade term $\delta E = \alpha_2 e^2 / \kappa r_{ij}$ ~\cite{aharony92}. The Differences can be summarized in a single plot ~\ref{MvsES} ~\cite{Liu10}.

\begin{figure}[htbp]
\begin{center}
\includegraphics[scale=.50]{MottvsES.png}
\caption{Density of states vs Temperature. This plot describes the interface between ES and Mott transmission of electrons. (Graph courtesy of Heng Liu)}
\label{MvsES}
\end{center}
\end{figure}

\subsection{Coulomb Glass vs Artificial Nanosolids}
Coulomb glasses are the original substrate on which the ES model of electron conduction was based. The "glass" in coulomb glass refers to a phase in which electron-electron interactions impair conduction and dynamics become slow, like the flow of a glass~\cite{ortuno04}. Artificial nanosolids are arrays of granules which have a higher intra-granule conduction than a inter-granule conduction. The differences begin at the density of states. In artificial nanosolids, the density of states can be more creative. By changing the size of the granule, and the material it is made out of (metal, insulator, superconductor) the amount of available electron slots per site (and the energy of each slot) can be chosen. There are two energy scales for each grain. $\delta$ is the mean level spacing. $E_c$, the energy required to pack on one more electron onto a site is on the order of 3000 Kelvin ~\cite{glatz08} .The distances are also different. For coulomb glasses, the distances are basically the inter-atomic distances. There we are basically talking about electron jumps between atoms on a crystal lattice. As the name suggests, in artificial nanosolids the distances are arbitrary. For typical artificial nanosolid applications, we are in the nanometer to tens of nanometers range~\cite{beloborodov05}. While the general electric potential plays a role in both systems, It plays a bigger role in the coulomb glass since the scales are smaller and the electric potential scales as $1/r$.

\subsection{Inelastic vs. Elastic tunneling}
As well as blocking each other, electrons also can impart energy onto each other which may result in an electron being dislodged. This is referred to as inelastic co-tunneling. If instead the electron tunnels or otherwise travels without dislodging, it is referred to as elastic co-tunneling (see fig. ~\ref{inelasticvselastic}). For low to medium temperatures, the main tunneling mechanism will be elastic. At higher temperatures, we expect a transition to in-elastic tunneling~\cite{Glazman05}.

\begin{figure}[htbp]
\begin{center}
\includegraphics[scale=.50]{inelasticvselastic.png}
\caption{a) elastic transmission of electrons. b) inelastic transmission of electrons.}
\label{inelasticvselastic}
\end{center}
\end{figure}

\subsection{Temperature}

Temperature-dependent phase transitions are scattered throughout the spectrum and so it is important to specify at which temperatures we are working with. At temperatures lower than the Neel temperature, We encounter the first transition called "Mott-Heisenberg". At these low temperatures, the magnetic fields of the electrons have a chance to couple into anti-ferromagnetic pairs. Also, During tunneling events, electrons will avoid atoms which are occupied with other electrons of similar spins as sharing an atom would require an increase in energy compared to one of opposite spin ~\cite{Gebhard03}. At medium temperatures, we have a transition between elastic and inelastic. In order for an electron to land on an occupied site it must have enough energy to overcome the Coulomb blockade. This is only possible if it was given enough kinetic energy from a phonon (sufficient thermal energy). As the temperature is increased, electrons can begin stacking. Once an electron stacks onto another electron, Coulomb repulsion guarantees that either electron will quickly move on~\cite{Glazman05}. At much higher temperatures, the system starts to have enough free thermal energy to exceed the coulomb blockade energy. That is, the energy required to stack 2 electrons in the same quantum dot. This yields a crossover temperature which was discussed during our comparison of ES vs Mott systems ~\cite{aharony92}. Depending on the density of states, there will eventually be higher energy states on which to pack on electrons. For our purposes, we will stay in between these two systems. Our temperatures will be high enough that magnetic moments can be ignored, yet low enough that the maximum number of electrons allowed on the same quantum dot will be two.

\subsection{Density of States}
There are many methods of characterizing the energy of an electron. One of these is the density of states. This is a histogram of the energies that electrons can be in. For example in ~\ref{moundDoS}, there are many electrons at a relatively low energy and the number of electrons with higher energies decreases with energy. In ~\ref{crystalDoS} there is only one energy option for electrons. We see two peaks because the holes act as particles in that the system requires energy to move those as well. The value on the x axis is the energy required to fill or empty each hole or particle respectively. By knowing the shape of the density of states, one can discern important intricacies of a system. These can include the energy scales, the relaxedness, and the bandgap of a system. The typical kinetic energy of an electron in a fermi glass is $E = E_o + \frac{(\hbar k)^2} {2m}$. If one follows through the math, the dispersion relation ends up as $D_n(E) = \frac {nc_n} {p c^{n/p}_k} (E- E_o)^{n/p-1}$ ~\cite{Kittel96}. If the particle distribution is symmetric (same amount of electrons as holes) then the density of states will be symmetric, as long as the system is relaxed.

\begin{figure}[htbp]
\begin{center}
\includegraphics[scale=.50]{veryCloseDos.png}
\caption{The density of states for a system with a high amount of randomness in the energy.}
\label{moundDoS}
\end{center}
\end{figure}

\begin{figure}[htbp]
\begin{center}
\includegraphics[scale=.50]{splitDos.png}
\caption{A low entropy system (Wigner crystal).}
\label{crystalDoS}
\end{center}
\end{figure}

		% file with references
\chapter{Geometric Vortex Pinning}
\label{theoryvortex}

\section{London Superconductivity}
Superconductivity was first explained throught he light of Maxwell's equations. In 1935, Fritz and Heinz London used their equations to explain the Meissner effect. these can be derived from Ampere's law:
\begin{eqnarray}
\nabla \times \overrightarrow B  = \mu_0 \overrightarrow J,
\label{Ampere}
\end{eqnarray}
where $\overrightarrow B$ is the magnetic field, $\mu_0$ is the permeability of free space, and $\overrightarrow J$ is the current. We then use the vector identity

\begin{eqnarray}
\nabla \times \nabla \times \overrightarrow A  = -\nabla^2 \overrightarrow A,
\label{stokes}
\end{eqnarray}
to get
\begin{eqnarray}
\nabla^2 B = \frac{B}{\lambda_p^2},
\label{penetration}
\end{eqnarray}

where $\lambda_p$ is the penetration depth. While these equations were good at describing the macroscopic properties of superconductors, better theories such as the Langevin system had to be derived to describe vortices. 

\section{Langevin system for vortices}
The Langevin system for vortices is a relatively simple place to start. It keeps only vortex degrees of freedom and foregoes all other perturbations of the complex order parameter. As long as the pinning centers are not too dense and vortices stay further appart than their coherence length, this provides a fairly accurate system. Looking at the equation of motion for this overdamped system, one can get an idea of the forces involved:
\begin{eqnarray}
\eta \frac {\partial u}{\partial t} = \epsilon_1 \frac {\partial^2 u} {\partial z^2 } + \Sigma_j F_{vp} (u - R) \delta(z - z_j) + j + F_T(z,t), 
\label{Langevin}
\end{eqnarray}
where $\eta$ is the viscosity coefficient, $\epsilon$ is the line tension, $f$ is the current's driving force, $(R,Z)$ are the random pinning coordinates, $F_{vp}$ is the pinning force, and $F_T$ is the thermal randomizing force. From this equation, one can see that the current driving force is trying to push overdamped vortices through a viscous fluid. Every once in a while, they get stuck in pinning centers and may or may not get out again depending on the current force and the thermal noise. Finally if the system is in 3 dimensions, then the vortices will want to stay in as straight of a line as possible~\cite{Kwok16}. As simple and intuitive as the Langevin system may be, it falls short on explaining other important phenomena such as vortex creation, cutting, and reconnection. Also, it cannot explain phenomena related to $T_C$ variation. For these we need a much more powerful system called the time dependent Ginzburg Landau model. 

\section{The Ginzburg-Landau Model for Superconductivity}
Superconductivity near the transition temperature can be succinctly characterized by a complex order parameter field $\psi$. Near this critical temperature, the free energy of the system becomes
\begin{eqnarray}
F = F_n + \alpha |\psi|^2 + \frac {\beta} {2} |\psi|^4 + \frac {1} {2m} |(-i \hbar \nabla - 2 e \overrightarrow A) \psi|^2 + \frac {|\overrightarrow B |^2} {2 \mu_0}
\label{freeE}
\end{eqnarray}
where $F_n$ is the free energy, $\alpha$ and $\beta$ are system constants which will be defined later,$\mu_0$ is the magnetic permeability of free space , $m$ is the cooper pair effective mass, $e$ is electron charge, $\overrightarrow A$ is the magnetic vector potential, and $\overrightarrow B$ is the magnetic field. Being interested in the physical reprecussions of this kind of system, we minimize the free energy with respect to the order parameter and vector potential to get

\begin{eqnarray}
\alpha \psi + \beta |\psi|^2 \psi + \frac {1} {2m} (-i \hbar \nabla - 2 e \overrightarrow A)^2 \psi = 0
\label{GLEQ1}
\end{eqnarray}
\begin{eqnarray}
\nabla \times \overrightarrow B = \mu_0 \overrightarrow j
\label{GLEQ2}
\end{eqnarray}
\begin{eqnarray}
\overrightarrow j = \frac {2e} {m} Re(\psi^* (-i \hbar \nabla - 2 e \overrightarrow A) \psi)
\label{GLEQ3}
\end{eqnarray}
where $j$ is the current and $Re$ is the real part. Equation ~\ref{GLEQ1} can be thought of in two parts. The first part ($\alpha \psi + \beta |\psi|^2 \psi $) is just the component relating to the superconductor, without supercurrent. Above the superconducting temperature, only $\psi = 0$ solves the equation. Below the superconducting temperature we have $|\psi|^2 = -\frac {\alpha} {beta}$, which is reminiscent of a quantum observable. The second half of ~\ref{GLEQ1} is basically a modified version of the Schrodinger time-independent equation, but with a magnetic potential. ~\ref{GLEQ2} is Ampere's law. ~\ref{GLEQ3} is also similar to a quantum observable with $ -i \hbar \nabla - 2e \overrightarrow A$ as the momentum operator in the presence of a magnetic field. The Ginzburg-Landau equations can be related to the microscopic Bardeen-Cooper-Schrieffer theory ~\cite{Sadovskyy14}. 

The SciDAC project revolves around a program called {\sc GLGPU} which models the important parameter function $\psi$ using the time dependent Ginzburg Landau equations:
\begin{eqnarray}
\Gamma (\partial_t +i \frac{2e}{\hbar}\mu)\psi = a_0 \epsilon (r) \psi - b |\psi|^2 \psi + \frac{1}{4m} (\hbar \Delta + \frac{2e}{ic} A)^2 \psi + \xi (r,t)
\label{TDGL1}
\end{eqnarray}
and
\begin{eqnarray}
\kappa^2 \nabla \times (\nabla \times A) = J_n + J_s + I,
\label{TDGL2}
\end{eqnarray}
where $u = \Gamma/(a_0 t_0)$, $t_0$ is the unit of time, and $J$ is the total current density. This program models a discretized system which is initialized with a parameter function and then each timestep is defined by following the TDGL equation. Using this program, we studied many situations such as grids of non-superconducting inclusions, funnels, vortex ratchets, and critical currents.


\section{Large $\lambda$ Limit}
The surface of a superconducting field is a rich interplay between the coherence length $\xi = \sqrt{\hbar^2/4ma_o}$ and the magnetic penetration length $\lambda = \sqrt{mc^2 / 8\pi e^2 \psi^2_0}$, where $\psi = \sqrt{a_0/b_0}$ is the equilibrium value of the order parameter in the absence of an electromagnetic field. The ratio $\chi = \lambda / \xi$ is called the GL parameter. 

 In high temperature superconductors, the penetration length $\lambda$ is typically much larger than the coherence length $\xi$. One consequence of this is that at length scales smaller than $\lambda$, the magnetic field is basically constant. The coherence length comes from the fermi velocity for the material and the energy gap between conducting and superconducting states~\cite{Kittel96}.  $\xi$ describes the length over which the superconducting electron density can change. This is also the size to which a cooper pair can spread. Another place it comes in is determining the size of the core of the vortex. 

\section{Phenomenological Vortex Interactions}

\subsection{Vortex-Vortex Interactions}
	 Vortex flux lines will move due to Lorentz forces $W = d \cdot F = d \cdot (B \times I)$ where $W$ is the energy drained away by moving vortices, $d$ is the distance the vortex travels, $F$ is the force on the vortex, $B$ is the field of the vortex, and $I$ is the current of the system. Whether or not a vortex will move is complicated and hard to study analytically. There are many factors which affect the movement of a vortex. First, vortices which rotate in the same direction will push each other away~\ref{sameV}. Conversely, if the vortices spin in the same direction, they will attract and annihilate each other~\ref{diffV}. This can be explained if one looks at the lorentz forces of a vortex-vortex system.

\begin{figure}[htbp]
\begin{center}
\includegraphics[scale=.50]{sameVortex.png}
\caption{The right hand rule tells us which direction the currents will swirl around a vortex pointing into the page. Then using Lorentz's law  at the second vortex, we can see that the second vortex will be pushed away. The same thing is happening from the current of the second vortex onto the first.}
\label{sameV}
\end{center}
\end{figure}

\begin{figure}[htbp]
\begin{center}
\includegraphics[scale=.50]{oppositeVortex.png}
\caption{The right hand rule tells us which direction the currents will swirl around a vortex pointing into the page. Then using Lorentz's law  at the second vortex, we can see that the second vortex will be pulled in this time. The same thing is happening from the current of the second vortex onto the first (although the second current flows in the opposite direction).}
\label{diffV}
\end{center}
\end{figure}

\subsection{Vortex-Defect Interactions}
The interaction between vortices and defects have been well-studied. Defects can be used to pin vortices and reduce dissipation. Defects can be impurities, vacancies, and inclusions. In one dimension they can be defects such as dislocations and irradiation tracks. Finally in three dimensions, they can be twin boundaries or stacking faults. The structure of atomic defects can have scattering properties which are either potential or magnetic type. Cooper pair-breaking caused by these defects has the effect of lowering the critical temperature. While studying these is important for pinning dynamics, it is the defecs which are on the order of several coherence lengths which are of interest to us. In three dimensions one can create columnar defects in the direction of the expected magnetic field. These have the strongest ability to hold on to vortices while causing the least amount of obstruction ~\cite{Kwok16}.

\subsection{Vortex-Wall Interactions}
The substrate is not always superconducting. It is also possible to embed non-superconducting components into the system. Since the super-current is suppressed in this system, There are many interesting effects to observe near these zones. First, a vortex near a magnetically neutral non-superconducting zone will be attracted to it. This can be seen as a type of "Venturi effect". Since the current flux must be conserved around a vortex, and there is not as much room for it to flow on the side closest to the wall, the current increases. A difference in current flux on opposing sides of the vortex then creates a force in the direction of the wall. The vortex is then pulled into the wall. The vortex will tend to remain there until pushed out by an external force. Two of the most common forces are external applied current and other vortices.

\begin{figure}[htbp]
\begin{center}
\includegraphics[scale=.50]{geometryInside.png}
\caption{Vortices (pink) will tend to get stuck inside inclusions (blue). That is because the supercurrent will travel faster outside of the inclusion, diminishing the amount of supercurrent inside. The vortex will therefore feel a force away from the boundary with the stronger superconductor.}
\label{geometryInside}
\end{center}
\end{figure}

\begin{figure}[htbp]
\begin{center}
\includegraphics[scale=.50]{repulsion.png}
\caption{Once a vortex is stuck inside an inclusion, it will repel any other vortices which come near it.}
\label{repulsion}
\end{center}
\end{figure}

\section{Relevant Vortex studies}

Type 2 superconductors, while working at liquid nitrogen temperatures, have the problem of energy dissipation due to vortex dynamics. Therefore many geometric solutions are being worked on to try to slow down this energy loss. Here we focus on two of the most promising, vortex matching and funneling.

\section{Vortex Matching and $\delta T_c$ Pinning}
With improvements in nanotechnology, it has become easier to create grids of inclusions. These can be made by inserting nanorods of different critical superconducting temperature ($T_c$), pulsed laser deposition , by using ion beams to disrupt the superconducting lattice, or by using lithographic techniques.

Inserting nanorods of materials using bulk targets has is benefits and problems. These nanorods can be made to have different $T_c$ by controling the crystalinity or the amount of oxygen doping. \cite{Horii15} found a positive vortex-pinning effect due to the nanorods up to a magnetic field of 5 Tesla. They calculated the inter-vortex distance and designed a particular nanorod density. From there they were able to show that vortex-matching had a stronger pinning effect on the system. Varying the $T_c$ of the system is a powerful way to control the vortex dynamics.

Pulsed laser deposition of YBCO can extend the irreversibility of magnetic fields past the 10 Tesla marker. This was done by using BYTO additions to pin the fluxes. All of this is thanks to the increased availability of ordered nanosized oxide secondary phases in epitaxial thin films. This allows the tuning of material functionalities and has various applications including high temperature superconductors. They observed a large improvement to the critical current compared to untouched YBCO~\cite{Rizzo16}. Accurate placement of inclusions is important as inter-vortex forces are on the order of $1/r$.      

Ion beam etching can also be used to create the necessary inclusions. These antidot arrays can be made on thin films of Niobium Nitride using reactive dc sputtering. The antidot arrays are then created using a mask-aligner to do the ion-beam etching. They find experimental evidence for the observation that the maximum number of vortices which can be captured by an antidot of diameter d is
\begin{eqnarray}
n_s = \frac {d} {4 \xi(t)}, 
\label{}
\end{eqnarray}
where $\xi(t) = \xi_0 / \sqrt{1-t}$ and $t$ is a reduced temperature, typically in the range of $0.9 - 0.95$. They also found that vortices would become trapped in the antidots as well as in the interstices of the antidot lattice. This means that the critical current depends on the geometry of the lattice as well as the direction of current~\cite{Thakur09}. Multiple vortex antidots and inter-vortex trapping are important phenomenta when studying vortex matching. A theoretical basis for vortex matching has also been found. Berdiyorov et. al. ~\cite{Berdiyorov06} used the nonlinear Ginzburg-Landau theory to obtain all configurations for vortices in a grid of defects. They also find that vortices will pin in the inclusions and in between them. For small inclusions, they find only one vortex is trapped per hole. The hole radius and inter-hole distance determines the ability of multiple vortices to be forced into the holes. Like spheres in a box, The vortices prefer a triangular lattice as that affords the densest packing. If the pinning force of the holes becomes small enough, the lattices shift from the grid imposed quadratic lattice to a more natural triangular lattice. They find matching effects at whole and fractional magnatic field to hole ratios. Finally, they found their results to not agree with the Little-Parks theory of superconductors.

Lithographic (sputter etching) techniques were used to create arrays of submicrometere sized pinning arrays. These were compared to simulated $J_c$ curves using the time dependent Ginzburg Landau model. They found that the critical current exibited maxima at the expected matching fields at 2.3 degrees Kelvin. The critical current was considerably larger than systems without antidot arrays~\cite{Sabatino10}.
It is important to have a physical temperature at which these phenomena occur as most Ginzburg-Landau simulations only describe temperature in relation to $T_c$.

\section{Vortex Jamming and Geometric Pinning}
Another way to increase the depinning current of a system is to get the vortices to pin each other. This jamming effect can be accomplished by manipulating the geometry of the system they must past through. Different geometric strategies have been tried including simple funnels, diamonds, and conformal maps. Vortex ratchets and fluxon pumps are also of great interest in the superconducting world. These use diode-like geometries to keep vortices going in one direction. These effects can be seen even with a symmetric force such as an alternating current.  

Computer simulations have been popular in demonstrating the usefulness of vortex ratchets. Within these ratchets, vortices will go through certain phases depending on the strength of the magnetic field. These phases are known as triangular, smectic, disordered, and square~\cite{Lu06}. They showed that sawtooth ridges which modulate the z-component of a superconductor can have a similar effect as triangular modulations in the x and y-components. They used a Langevin model to simulate the vortices, which starts to become unphysical as the vortex density increases.  

The optimal size of the funnel tip is such that only one vortex can pass at a time due to vortex-vortex repulsion forces. Reichhardt et. al.~\cite{Reichhardt10} , through the help of simulations, observed that the sum of the vortex velocities remains constant with increasing magnetic field. They highlight the similarities to a granular hopper, decreasing the width of the hopper aperture decreases the flow of grains. They found that as the number of vortices increases, the pinned vortex structure becomes larger and harder to deform. This then keeps individuals from passing through the bottleneck.

Fluxon pumps can be used to extract useful work from a fluctuating environment. The vortex ratchet can be used as a fluxon rectifier. These could be used as fluxon lenses to concentrate or disperse magnetic fields. These would have various uses including dispersing trapped flux in SQUID magnetometers. But to get the desired effects, the frequency must hit the appropriate resonance region~\cite{Wambaugh99}. This means not only the right frequency for the geometry, but also the correct temperature (i.e. random motion). Thermal noise had to be kept low in the pinning systems. Like sand in an hourglass, minor turbulence has a large effect on the ability of vortices to stick. They again back up the idea that the more fluxons one has, the more their motion will be restricted.

Vortices travelling through constricting lattices can also be used to study the dynamics of interacting particles travelling through confining potentials. Yu et. al. found that with the correct geometry, vortices would begin moving as the external magnetic field varied~\cite{Yu10}. They found strong matching effects between the vortex distribution and the constriction lattice. By tailoring their diamond shaped channel, they could pick their confining potential. The angle of the wall has a strong effect on the ability of the system to be immobilized.

We built upon previous GLGPU research in geometric pinning. In order to simulate geometrical constraints and voids of various shapes we impose appropriate internal boundary conditions or use unstructured grid discretizations~\cite{Kwok16}. The boundary conditions are imposed via picking no current (open) boundaries which simulate insulating inclusions~\cite{Sadovskyy14}. Multiple types of tesselations are also possible, including checkerboard, or less standard Voronoi tesselations. These are useful for polycrystalline thin superconducting films with variations of $T_c$.

\section{GLGPU overview}
GLGPU is a GPU-based Jacobi solver for the time-dependent GL equations. It was designed to solve mesoscale problems which arised in superconductor design. Instead of treating the vortices as elastic strings in a viscous medium, it focuses on the underlying order parameter. This allows for correct interactions between pairs of vortices, vortices and inclusions, and allows the vortices to split up and rejoin. To model the system accurately, we need to take into account the Ginzburg Landau function at a microscopic resolution as a complex valued scalar field. The amplitude of this field is related to the supercondonductivity density, while the phase is related to the current of the system (after a gauge transformation). Through careful optimization, the system size can be made large enough to encompass many vortex interactions and complicated non-superconducting architectures. Physical pinning defects are simulated as modulations of the superconductor's critical temperature. The simulation works by first integrating the GL equations forward in time, then solving the Poisson equiation to find the electric and magnetic fields. There are also noise correlation terms to simulate thermal effects~\cite{Sadovskyy14}.

\subsection{stable GLGPU system}
The particular Ginzburg Landau equations which were used are designed to be particularly stable. In appendix 1, we explore the limits of this system and find it works for all relevant currents. These current equations were :
\begin{eqnarray}
J_N = -\sigma [(1/c) \partial_t A + \nabla \mu ]
\label{currentEq1}
\end{eqnarray}
and
\begin{eqnarray}
J_S = -\frac{e}{2m} [\psi*(i\hbar \nabla + \frac{2e}{c} A) \psi + c.c.],
\label{currentEq2}
\end{eqnarray}
where $\nabla A = 0$, $\sigma$ is the conductivity, $\mu$ is the scalar potential, $A$ is the vector potential, $e$ is electron charge, $c$ is the speed of light, and $m$ is electron mass. To help with stability, the Crank-Nicolson integration scheme and the linearization of the $|\psi|^2\psi$ term were done. An explicit integration would be subject to numerical instabilities.

\subsection{Boundary conditions}

There are two types of boundary conditions relevant to this system. They are quasi-periodic and open. On a quasi-periodic boundary, cells are treated the same as they would be in the middle of the system. The only thing that one needs to keep track of is the phase jump at the boundary. In other words, only the amplitude of the order parameter is really periodic. The open boundary refers to the Neumann boundary conditions. These mean that there is no current perpendicular to the boundary. Specifically;
\begin{eqnarray}
\overrightarrow n (\nabla - i \overrightarrow A)\psi = 0 ,
\lable{}
\end{eqnarray}
where $\overrightarrow n$ is the unit normal vector. For our studies, we stuck with quasi-periodic boundary conditions in the direction of current.



\section{Results}
Being such a general code, there were many possible areas of study regarding this code. We found promissing results in two general areas, grids of inclusions and super conducting-conducting funnels with higher $T_c$. We characterize these situations by looking at their responses to changes in voltage, magnetic field, and geometry. There is an important choice that we made when determining the critical current as there is a inherent hysteresis in the system. We had two options when ramping the current. It could either have been ramped up, Starting with no movement and pushing the current until the vortices became dislodged. The second option was to ramp the current down. That is start with moving vortices and slowly decrease the current until the vortices become lodged. In the end we went for the second option due to the dynamics of the system. Vortices tend to start out placed not according to the lowest energy state, but instead randomly according to whatever starting seed was selected. By forcing the vortices to move around first, we get a more natural state before the depinning current is found. In figure ~\ref{hysteresis} we quantitatively demonstrate the difference.

\begin{figure}[htbp]
\begin{center}
\includegraphics[scale=.50]{JvV.png}
\caption{ Overlapped are the current vs voltage for two systems, one with the current ramped up and one with the current ramped down. Upper one is the current being ramped down, while the lower one is the current ramped up.}
\label{hysteresis}
\end{center}
\end{figure}


\subsection{Grids}
Inclusions in the superconductive substrate can be used to contain vortices (up to a certain current). If these vortices do not move, then they are not draining energy from the system. The first thing we needed to show was that vortices preferred to be trapped in inclusions which were around the same size as they were. If the inclusion size is increased, we see that at some point we start to trap 2 vortices per inclusion. Trapping a vortex requires that the force due to external current and other vortices be less than the superconducting destruction force. The superconducting destruction force is due to the energy of a superconducting system below Tc Being lower than a non-superconducting system below Tc. The second effect we were looking for was vortex matching resistance. Vortices have quantized magnetic flux. This means that for a certain applied field, we can predict the number of vortices. If some vortices are moving and some are pinned, the moving ones will push on the pinned ones as they pass by. If instead we have as many vortices as we have inclusions, after a certain relaxation period, vortices will all be pinned and the system will be more stable. As shown in Fig.~\ref{HDF} we were able to show both of these effects. In the X direction, we see that the optimal radius for an inclusion is 2. In the Y direction, we see a sharp drop in critical current at m=1. This is because as soon as we are above the 1-to-1 ratio, we start to have rogue vortices which will move. As the magnetic field is increased, we start to see that there islands of stability where multiple vortices can fit on the same inclusion. There is also a quadratic relation between the magnetic ratio and the size of the vortices in terms of optimal conditions.

\begin{figure}[htbp]
\begin{center}
\includegraphics[scale=.50]{HDFinal.png}
\caption{ This is a superposition of 100+ simulations with different parameters. The critical current of each simulation was shown using color. On the X-axis is the radius of each inclusion in the system. On the Y axis m is the ratio of number of vortices per inclusion. The colorbar stands for critical current. In other words, the redder a zone is, the better it holds on to vortices, the higher the current can be before the vortices begin to slip.}
\label{HDF}
\end{center}
\end{figure}

\section{Funnels}
The same way that energy is gained when a vortex goes into a lower superconducting state, energy is lost if it tries to go into a higher superconducting state. Also, since more current can travel through the superconducting system, we get the opposite of the venturi effect. This also helps to keep vortices out. Using this logic, we created superconducting zones with higher $T_c$ that can be used to guide vortices. Indeed, increasing the aperture size increases the ease of which the vortices can travel ~\ref{AvR}.

\subsection{Description}
The magnetic field was kept such that the number of vortices filled 50\% to 80\% of the substrate. It was pointed into the board such that the current pushed the vortices into the funnel. The current was started at an amount large enough to guarantee vortex movement, even if it also meant that vortices went through the walls at first. The current was then slowly relaxed in enough timesteps to make sure that each configuration had enough time to stabilize. The number of current configurations had to balanced against practicallity. Ideally, we would use an infinite number of current configurations, but that would take an infinite ammount of time to simulate. Instead we tried to keep the simulation times at less than 24 hours.

\subsection{Analysis}
In general, the simulations performed as expected. The smaller the aperture, the better a system can hold on to vortices. This is up to a point which where even 1 vortex can no longer slip through. Steeper angles also improved the jamming performance of each system. The dependence did not seem to be linear in y-position of the funnel, or in angle. When looking at the magnetic dependence of the critical current, it is clear that more vortices increase the critical current of the system.
\begin{figure}[htbp]
\begin{center}
\includegraphics[scale=.50]{ratchetNoAngle.png}
\caption{ The amplitude of the complex order parameter. in yellow is the background superconductor, In red is the superconductor wall, and the blue dots are the vortices. In green is the parameter of interest. In this case, we have a flat obstruction which forces vortices through a narrow gap.}
\label{noAngle}
\end{center}
\end{figure}

\begin{figure}[htbp]
\begin{center}
\includegraphics[scale=.50]{flatScan.png}
\caption{ 50 values of aperture were run in this simulation. The resulting current was analyzed to find the critical current . As the aperture is increased, the ability to hold the vortices still is diminished. }
\label{flatScan}
\end{center}
\end{figure}


\begin{figure}[htbp]
\begin{center}
\includegraphics[scale=.50]{normalX.png}
\caption{ The amplitude of the complex order parameter. in yellow is the background superconductor, In red is the superconductor wall, and the blue dots are the vortices. In green is the parameter of interest. In this case it is the size of the aperture which was varied. }
\label{normalX}
\end{center}
\end{figure}

\begin{figure}[htbp]
\begin{center}
\includegraphics[scale=.50]{normalXscan.png}
\caption{ 50 values of aperture were run in this simulation. The resulting current was analyzed to find the critical current . As the aperture is increased, the ability to hold the vortices still is diminished. }
\label{normalXscan}
\end{center}
\end{figure}

\begin{figure}[htbp]
\begin{center}
\includegraphics[scale=.50]{normalY.png}
\caption{ The amplitude of the complex order parameter. in yellow is the background superconductor, In red is the superconductor wall, and the blue dots are the vortices. In green is the parameter of interest. In this case it is the point on the Y-axis at which the funnel attaches and therefore the angle which varies. }
\label{normalY}
\end{center}
\end{figure}

\begin{figure}[htbp]
\begin{center}
\includegraphics[scale=.50]{normalYscan.png}
\caption{ 50 values of aperture were run in this simulation. The resulting current versus voltage information was analyzed to find the critical current . As the angle of the slope is increased, the vortices were more easily jammed. }
\label{normalYscan}
\end{center}
\end{figure}


\begin{figure}[htbp]
\begin{center}
\includegraphics[scale=.50]{normalAngle.png}
\caption{ The amplitude of the complex order parameter. in yellow is the background superconductor, In red is the superconductor wall, and the blue dots are the vortices. In green is the parameter of interest. In this case the angle is held constant while the x and y attachments are varied. }
\label{normalAngle}
\end{center}
\end{figure}

\begin{figure}[htbp]
\begin{center}
\includegraphics[scale=.50]{constantAngle.png}
\caption{ This is a scan of the parameter of interest versus the critical current. The x-postition and y-position of the funnel were moved uniformly as to keep the angle constant. }
\label{constantAngle}
\end{center}
\end{figure}


\begin{figure}[htbp]
\begin{center}
\includegraphics[scale=.50]{oneSidedDone.png}
\caption{ The amplitude of the complex order parameter. in yellow is the background superconductor, In red is the superconductor wall, and the blue dots are the vortices. In green is the parameter of interest. In this case it is the size of the aperture which was varied. }
\label{oneSidedX}
\end{center}
\end{figure}


\begin{figure}[htbp]
\begin{center}
\includegraphics[scale=.50]{oneSidedAperture.png}
\caption{ 50 values of aperture were run in this simulation. The resulting current versus voltage information was analyzed to find the critical current . As the aperture size is increased, the vortices are less restrained. }
\label{normalYscan}
\end{center}
\end{figure}

\begin{figure}[htbp]
\begin{center}
\includegraphics[scale=.50]{oneSidedY.png}
\caption{ The amplitude of the complex order parameter. in yellow is the background superconductor, In red is the superconductor wall, and the blue dots are the vortices. In green is the parameter of interest. In this case it is the point on the Y-axis at which the funnel attaches and therefore the angle which varies.}
\label{oneSidedY}
\end{center}
\end{figure}

\begin{figure}[htbp]
\begin{center}
\includegraphics[scale=.50]{oneside-angle-Scan.png}
\caption{ 50 values of aperture were run in this simulation. The resulting current versus voltage information was analyzed to find the critical current . As the one slope is increased, the jamming effect becomes more pronounced. }
\label{normalYscan}
\end{center}
\end{figure}

\begin{figure}[htbp]
\begin{center}
\includegraphics[scale=.50]{oneKinkDone.png}
\caption{ The amplitude of the complex order parameter. in yellow is the background superconductor, In red is the superconductor wall, and the blue dots are the vortices. In green is the parameter of interest. In this case it is the length of the "turn" which is being varied.}
\label{oneSidedY}
\end{center}
\end{figure}

\begin{figure}[htbp]
\begin{center}
\includegraphics[scale=.50]{kinkScan.png}
\caption{ 50 values of the turn length were run in this simulation. The resulting current versus voltage information was analyzed to find the critical current .  }
\label{normalYscan}
\end{center}
\end{figure}



\begin{figure}[htbp]
\begin{center}
\includegraphics[scale=.50]{AvR.png}
\caption{ We can also study how the aperture size hampers vortex movement once they have already been depinned. This is a resistance plot of the constant y-position funnel. On the X axis is the size of the aperture, Y axis is the superconductive resistance. }
\label{AvR}
\end{center}
\end{figure}

\begin{figure}[htbp]
\begin{center}
\includegraphics[scale=.50]{angleBz.png}
\caption{ 50 values of the magnetic field angled system. The resulting current versus voltage information was analyzed to find the critical current .  }
\label{angleBz}
\end{center}
\end{figure}


\begin{figure}[htbp]
\begin{center}
\includegraphics[scale=.50]{emptyBz.png}
\caption{ 50 values of the magnetic field were studied for the empty diamond system. The resulting current versus voltage information was analyzed to find the critical current .  }
\label{emptyBz}
\end{center}
\end{figure}

\begin{figure}[htbp]
\begin{center}
\includegraphics[scale=.50]{flatBz.png}
\caption{ 50 values of the magnetic field were studied for the flat system. The resulting current versus voltage information was analyzed to find the critical current .  }
\label{flatBz}
\end{center}
\end{figure}

\begin{figure}[htbp]
\begin{center}
\includegraphics[scale=.50]{oneSideBz.png}
\caption{ 50 values of the magnetic field for the one sided system. The resulting current versus voltage information was analyzed to find the critical current .  }
\label{oneBz}
\end{center}
\end{figure}

To conclude, we have explored multiple systems with the use of the GL program. We showed that if the number of inclusions matches the number of vortices, the system jams and we have much lower resistance due to vortex motion. We have also explored many funnel systems which are currently being tried out experimentally throughout the superconduction community. We showed that vortex jamming is a function of the angle of the funn
el, the size of the aperture, and the amount of vortices.

 

%this chapter is deprecated and should be deleted

\chapter{GLcode}          % chapter 1
\label{codechap}

\section{GL}
GL is a GPU based Ginzburg-Landau field simulator. It was designed to solve mesoscale problems which arised in superconductor design. To model the system accurately, we need to take into account the Ginzburg Landau function at a microscopic resolution. At the same time, the system size has to be large enough to encompass many vortex interactions and complicated non-superconducting architectures. Physical pinning defects are simulated as modulations of the superconductor's critical temperature. The simulation works by first integrating the GL equations forward in time, then solving the Poisson equiation to find the electric and magnetic fields. There are also noise correlation terms to simulate thermal effects~\cite{Sadovskyy14}.

\section{Results}
Being such a general code, there were many possible areas of study regarding this code. We found promissing results in two general areas, grids of inclusions and hyper-conducting funnels. We characterize these situations by looking at their responses to changes in voltage, magnetic field, and geometry.

\subsection{Grids}
Inclusions in the superconductive substrate can be used to contain vortices (up to a certain current). If these vortices do not move, then they are not draining energy from the system. The first thing we needed to show was that vortices preferred to be trapped in inclusions which were around the same size as they were. If the inclusion size is increased, we see that at some point we start to trap 2 vortices per inclusion. Trapping a vortex requires that the force due to external current and other vortices be less than the superconducting destruction force. The superconducting destruction force is due to the energy of a superconducting system below Tc Being lower than a non-superconducting system below Tc. The second effect we were looking for was vortex matching resistance. Vortices have quantized magnetic flux. This means that for a certain applied field, we can predict the number of vortices. If some vortices are moving and some are pinned, the moving ones will push on the pinned ones as they pass by. If instead we have as many vortices as we have inclusions, after a certain relaxation period, vortices will all be pinned and the system will be more stable. As shown in Fig.~\ref{HDF} we were able to show both of these effects. In the X direction, we see that the optimal radius for an inclusion is 2. In the Y direction, we see a sharp drop in critical current at m=1. This is because as soon as we are above the 1-to-1 ratio, we start to have rogue vortices which will move.

\begin{figure}[htbp]
\begin{center}
\includegraphics[scale=.50]{HDFinal.png}
\caption{ On the X-axis is the radius of each inclusion in the system. On the Y axis m is the ratio of number of vortices per inclusion. The colorbar stands for critical current. In other words, the redder a zone is, the better it holds on to vortices, the lower superconductive resistance it puts out.}
\label{HDF}
\end{center}
\end{figure}
 

\subsection{Funnels}
The same way that energy is gained when a vortex goes into a lower superconducting state, energy is lost if it tries to go into a higher superconducting state. Also, since more current can travel through the hyperconducting system, we get the opposite of the venturi effect. This also helps to keep vortices out. Using this logic, we created hyperconducting zones that can be used to guide vortices. Indeed, increasing the aperture size increases the ease of which the vortices can travel ~\ref{AvR}. 



\begin{figure}[htbp]
\begin{center}
\includegraphics[scale=.50]{AvR.png}
\caption{ On the X axis is the size of the aperture, Y axis is the superconductive resistance. }
\label{AvR}
\end{center}
\end{figure}


\begin{thebibliography}{10}
\bibitem{mott72} Mott, N. F.; Peierls, R. (1937). "Discussion of the paper by de Boer and Verwey". Proceedings of the Physical Society 49 (4S): 72

\bibitem{efros75} A L Efros and B I Shklovskii , J. Phys. C8, L49 (1975)

\bibitem{glazman05} L.I. Glazman, M. Pustilnik, in "Nanophysics: Coherence and Transport," eds. H. Bouchiat et al. (Elsevier, 2005), pp. 427-478

\bibitem{kirkengen09} M. Kirkengen , J. Bergli, "Slow relaxation and equilibrium dynamics in a two-dimensional Coulomb glass: Demonstration of stretched exponential energy correlations" PHYSICAL REVIEW B 79, 075205 2009

\bibitem{aharony92} A. Aharony, Y. Zhang, M.P. Sarachik, "Universal Crossover in Variable Range Hopping with Coulomb Interactions", Physical Review Letters 1992

\bibitem{joung} D. Joungand, S. Khondaker, "Efros-Shklovskii variable range hopping in reduced graphene oxide sheets of varying carbon sp2 fraction", http://arxiv.org/pdf/1210.1876.pdf

\bibitem{glatz09} A. Glatz, I. Beloborodov, "Thermoelectric and Seebeck coefficients of granular metals", PHYSICAL REVIEW B 79, 235403 2009

\bibitem{chen} L. Chen, S. Gao, X. Zeng, A. Mehdizadeh Dehkordi,T. M. Tritt,and S. J. Poon, "Uncovering High Thermoelectric Figure of Merit in (Hf,Zr)NiSn Half-Heusler Alloys" , http://arxiv.org/pdf/1505.07773.pdf

\bibitem{ortuno04} M. Ortuno , A. Somoza, "Coulomb Glasses", http://www.lancaster.ac.uk/users/esqn/windsor04/docs/artuno-cg.pdf

\bibitem{beloborodov05} I. S. Beloborodov, A. V. Lopatin, F. W. J. Hekking, R. Fazio, and V. M. Vinokur , "Thermal transport in granular metals ",Europhys. Lett. 69, 435 (2005)

\bibitem{glatz08} A. Glatz ,I. S. Beloborodov , "Nanogranular Thermoelectrics" , http://arxiv.org/pdf/0810.5545.pdf

\bibitem{Gebhard03} B. Gebhard, "The Mott Metal-Insulator Transition: Models and Methods", Springer 2003

\bibitem{Glazman05} L. Glazman, M. Pustilnik, "Low-Temperature Transport Through A Quantum Dot", http://arxiv.org/pdf/cond-mat/0501007v2.pdf

\bibitem{Liu10} H. Liu, A. Pourret, P. Guyot-Sionnest, "Mott and Efros-Shklovskii Variable Range Hopping in CdSe Quantum Dots Films", ACS Nano, 2010, 4 (9), pp 5211-5216

\bibitem{Vinokur08} A. Glatz, V. Vinokur, J. Bergli, M. Kirkengen, Y. Galperin, "The Coulomb gap and low energy statistics for Coulomb glasses", Journal of Statistical Mechanics: Theory and Experiment, 2008

\bibitem{Kittel96} C. Kittel, "Introduction to Solid State Physics", Wiley, 1996

\bibitem{Beloborodov07} I. Beloborodov, A. Lopatin, V. Vinokur, K. Efetov, "Granular Electronic Systems", Reviews of Modern Physics, Volume 79, April-June 2007

\bibitem{Sadovskyy14} I. Sadovskyy, A. Koshelev, C. Phillips, D. Karpeev, A. Glatz, "Stable large-scale solver for Ginzburg-Landau equations for superconductors", Journal of Computational Physics, 09/2014

\bibitem{Ferrero14} E. Ferrero, A. Kolton, M. Palassini, "Parallel kinetic Monte Carlo simulation of Coulomb glasses" ,arXiv:1407.5026

\bibitem{Krauth06} W. Krauth, "Statistical Mechanics: Algorithms and Computations", Oxford University Press, Oxford, 2006, pp. 33–34

\bibitem{Newman99} M. Newman, G. Barkema, "Monte Carlo Methods in Statistical Physics", Oxford University Press, Oxford, 1999, pp. 41-42

\bibitem{Sparks16} T. Sparks, M. Gaultois, A. Oliynyk, J. Brgoch, B. Meredig, "Data mining our way to the next generation of thermoelectrics", Scripta Materialia, Volume 111, 15 January 2016, Pages 10–15

\bibitem{Chandra??} R. Chandra, "Mechanical vapour compression refrigeration", Refrigeration and Air Conditioning, New Delhi, India: PHI Learning. p. 3. ISBN 81-203-3915-0.

\bibitem{Brown63} J. Brown, "Thermodynamics of a Rubber Band", American Journal of Physics, 31 (5): 397–397, May 1963, doi:10.1119/1.1969535

\bibitem{Miszczak15} M. Miszczak, "GINZBURG-LANDAU SIMULATIONS OF NARROW SUPERCONDUCTING STRIPS" , Northern Illinois University, Department of Physics, 2015

\bibitem{Kittel96} C. Kittel, "Introduction to Solid State Physics", 7th Ed., Wiley, (1996)

\bibitem{Reichhardt10} O. Reichhardt, C. Reichhardt, "Commensurability, Jamming, and dynamics for vortices in funnel geometries" ,Physical Review B 81, 2010

\bibitem{Haag14} L. Haag, G. Zechner, W. Lang, M. Dosmailov, M. Bodea, J. Pedarnig "Strong vortex matching effects in YBCO films with periodic modulations of the superconducting order parameter fabricated by masked ion irradiation", Physica C: Superconductivity, Volume 503, 2014, pg 75-81

\bibitem{Horii15} S. Horii, M. Haruta, A. Ichinose, T. Doi, "Evidence for enhancement of vortex matching field above 5 T and oxygen-deficient annuli around barium-niobate nanorods", Journal of Applied Physics 118, 133907 (2015)

\bibitem{Rizzo16} F.Rizzo, A.Augieri, A. Armenio, V. Galluzzi, A. Mancini, V. Pinto, A. Rufoloni, A. Vannozzi, M. Bianchetti, A. Kursumovic, J. MacManus-Driscoll, A. Meledin, G. Van Tendeloo, G. Celentano, "Enhanced 77 K vortex-pinning in Y Ba2Cu3O7−x films with Ba2Y TaO6 and mixed Ba2Y TaO6 + Ba2Y NbO6 nano-columnar inclusions with irreversibility field to 11 T", APL Mater. 4, 061101 (2016); doi: 10.1063/1.4953436

\bibitem{Thakur09} A. Thakur, S. Ooi, S. Chockalingam, J. Jesudasan, P. Raychaudhuri, K. Hirata, "Vortex matching effect in engineered thin films of NbN", APPLIED PHYSICS LETTERS 94, 262501 2009

\bibitem{Sabatino10} P. Sabatino, C. Cirillo, G. Carapella, M. Trezza, and C. Attanasio, "High field vortex matching effects in superconducting Nb thin films with a nanometersized
square array of antidots", Journal of Applied Physics 108, 053906 (2010)

\bibitem{Lu06} Q. Lu, O. Reichhardt, C. Reichhardt, "Reversible vortex ratchet effects and ordering in superconductors with simple asymmetric potential arrays", PHYSICAL REVIEW B 75, 2007

\bibitem{Wambaugh99} J. Wambaugh, C. Reichhardt, C. Olson, F. Marchesoni , F. Nori, "Superconducting fluxon pumps and lenses", VOLUME 83, NUMBER 24 PHYSICAL REVIEW LETTERS, 1999

\bibitem{Yu10} K. Yu, M. Hesselberth, P. Kes, B. Plourde, "Vortex dynamics in superconducting channels with periodic constrictions", PHYSICAL REVIEW B 81, 2010

\bibitem{Berdiyorov06} G. Berdiyorov, M. Milošević, F. Peeters, "Vortex configurations and critical parameters in superconducting thin films containing antidot arrays: Nonlinear Ginzburg-Landau theory", PHYSICAL REVIEW B 74, 2006

\bibitem{Mott68} N. Mott, "Conduction in glasses containing transition metal ions", Cambridge, Journal of non-crystaline solids, 1968

\bibitem{Glazman05} L. Glazman, M. Pustilnik,"LOW-TEMPERATURE TRANSPORT THROUGH A QUANTUM DOT" arXiv:cond-mat/0501007 v2 10 Oct 2005

\bibitem{Tsigankov02} D. Tsigankov, A. Efros, "Variable Range Hopping in Two-Dimensional Systems of Interacting Electrons", VOLUME 88, NUMBER 17 PHYSICAL REVIEW LETTERS 2002

\bibitem{Young66} W. Young and E. Elcock,"Monte Carlo studies of vacancy migration in binary ordered alloys", Proceedings of the Physical Society 89 (1966) 735.

\bibitem{Serebrinsky11} S. Serebrinsky, "Physical time scale in kinetic Monte Carlo simulations of continuous-time Markov chains", PHYSICAL REVIEW E 83, (2011)

\bibitem{Abeles75} B. Abeles, P. Sheng, M. Coutts, Y. Arie, "Structural and electrical properties of granular metal films", Advances in Physics, 1975

\bibitem{Hubbard09} D. Hubbard, D. Samuelson, "Modeling Without Measurements". OR/MS: 28–33. 2009

\bibitem{Kalos08} M. Kalos, P. Whitlock, "Monte Carlo Methods", Wiley-Blackwell, 2008

\bibitem{Everitt02} B. Everitt, "The Cambridge Dictionary of Statistics", CUP. ISBN 0-521-81099-X, 2002

\bibitem{Dommelen13} L. van Dommelen, "A.11 Thermoelectric effects". Eng.fsu.edu. 2002-02-01. Retrieved 2013-04-22.

\bibitem{Kwok16} W. Kwok, U. Welp, A. Glatz1, A. Koshelev1, K. Kihlstrom1, G. Crabtree, "Vortices in high-performance high-temperature superconductors", Reports on Progress in Physics, 2016

\bibitem{Rowe05} D. Rowe, "Thermoelectrics Handbook: Macro to Nano.", CRC Press. ISBN 978-1-4200-3890-3., 2005

\end{thebibliography}

%%%%%%%%%%%%%%%%%%%%%%%%%%%%%%%%%%%%%%%%%%%%%%%%%%%%%%%%%%%%%%%%%%%
%%  Appendices

\appendix
%\chapter{Dias Instructions}
\section{Input file}
Dias has a simple input file which allows the user to change parameters on the go. 
rejection- Good inputs are any positive number. This is to change the monte carlo system from a rejection free on (0) to a rejection one (1). 
timeName- Good inputs are character strings, typically ending with the ".txt" suffix. This is where the time of each jump can be pulled from 
boxName
lineName
whichBox
grabJ
relax
eV
L
tSteps
XYvar
muVar
Temp



Dias also has the option to load previous electron states.
			% file with Appendix A contents
%\chapter{Pushing the current limits of GL}

\section{Breakdown}
One of the first things that we did with the code was to push it to its limits. We did this by exploring its behavior near the current breakdown region. There we saw many interesting phenomena which were outlined in figure ~\ref{breakdown}.

\begin{figure}[htbp]
\begin{center}
\includegraphics[scale=.50]{breakdownBreakdown.png}
\caption{This is a plot of the voltage (resistance) of the system as the current is slowly ramped up. At low currents, we just see the main vortex stuck to the Tc inclusion. As the current is increased, we eventually reach the depinning current and the vortices begin moving. This can be seen in the phase diagram as the phase has begun to wind up, indicating vortex movement. Sometime phenomena that can be observed, is that anti-vortex/vortex pairs are created in the inclusion and propagate in opposite directions. In region 3, the vortices clearly wind up the phase diagram and the resistance increases dramatically. Eventually in region 4, the system destabilizes and we begin to get vortex/anti-vortex pairs from the substrate. Finally, in region 5, we get full breakdown due to current and the system is no longer superconducting.}
\label{breakdown}
\end{center}
\end{figure}

			% file with Appendix B contents

\end{document}
