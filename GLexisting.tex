\chapter{GLexisting}          % chapter 1
\label{codechap}
\section{Vortex studies}

\section{motivation}
Type 2 superconductors, while being great for working under liquid nitrogen temperatures, have the problem of vortex slipping. Therefore many geometric solutions are being worked on to try to slow down this energy loss. Here we focus on two of the most promising, vortex matching and funneling. 

\section{Vortex Matching}
With improvements in nanotechnology, it has become easier to create grids of inclusions. These can be made by inserting nanorods of different critical superconducting temperature ($T_c$), pulsed laser deposition , by using ion beams to disrupt the superconducting lattice, or by using lithographic techniques.

Inserting nanorods of materials using bulk RE123 targets has is benefits and problems. These nanorods can be made to have different $T_c$ by controling the crystalinity or the amount of oxygen doping. \cite{Horii15} found a positive vortex-pinning effect due to the nanorods up to a magnetic field of 5 Tesla. They calculated the inter-vortex distance and designed a particular nanorod density. From there they were able to show that vortex-matching had a stronger pinning effect on the system.

Pulsed laser deposition of YBCO can extend the irreversibility of magnetic fields past the 10 Tesla marker. This was done by using BYTO additions to pin the fluxes. All of this is thanks to the increased availability of ordered nanosized oxide secondary phases in epitaxial thin films. This allows the tuning of material functionalities and has various applications including high temperature superconductors. They observed a large improvement to the critical current compared to untouched YBCO~\cite{Rizzo16}.

Ion beam etching can also be used to create the necessary inclusions. These antidot arrays can be made on thin films of Niobium Nitride using reactive dc sputtering. The antidot arrays are then created using a mask-aligner to do the ion-beam etching. They find experimental evidence for the observation that the maximum number of vortices which can be captured by an antidot of diameter d is
\begin{eqnarray}
n_s = \frac {d} {4 \xi(t)}, 
\label{}
\end{eqnarray}
where $\xi(t) = \xi_0 / \sqrt(1-t)$ and $t$ is a reduced temperature, typically in the range of $0.9 - 0.95$. They also found that vortices would become trapped in the antidots as well as in the interstices of the antidot lattice. This means that the critical current depends on the geometry of the lattice as well as the direction of current~\cite{Thakur09}. 

Lithographic (sputter etching) techniques were used to create arrays of submicrometere sized pinning arrays. These were compared to simulated $J_c$ curves using the time dependent Ginzburg Landau model. They found that the critical current exibited maxima at the expected matching fields at 2.3 degrees Kelvin. The critical current was considerably larger than systems without antidot arrays~\cite{Sabatino10}.

A theoretical basis for vortex matching has also been found. Berdiyorov et. al. ~\cite{Berdiyorov06} used the nonlinear Ginzburg-Landau theory to obtain all configurations for vortices in a grid of defects. They also find that vortices will pin in the inclusions and in between them. For small inclusions, they find only one vortex is trapped per hole. The hole radius and inter-hole distance determines the ability of multi-quanta vortices to be forced into the holes. The vortices prefer a triangular lattice as that affords the most spacing in between the vortices. If the pinning force of the holes becomes small enough, the lattices shift from the grid imposed quadratic lattice to a more natural triangular lattice. They find matching effects at whole and fractional magnatic field to hole ratios. Finally, they found their results to not agree with the Little-Parks theory of superconductors.  

\section{Vortex Funnels}
Another way to increase the depinning current of a system is to get the vortices to pin each other. This jamming effect can be accomplished by manipulating the geometry of the system they must past through. Different geometric strategies have been tried including simple funnels, diamonds, and conformal maps. Vortex ratchets and fluxon pumps are also of great interest in the superconducting world. These use diode-like geometries to keep vortices going in one direction. These effects can be seen even with a symmetric force such as an alternating current.  

Computer simulations have been popular in demonstrating the usefulness of vortex ratchets. Within these ratchets, vortices will go through certain phases depending on the strength of the magnetic field. These phases are known as triangular, smectic, disordered, and square~\cite{Lu06}. They showed that sawtooth ridges which modulate the z-component of a superconductor can have a similar effect as triangular modulations in the x and y-components. 

The optimal size of the funnel tip is such that only one vortex can pass at a time due to vortex-vortex repulsion forces.~\cite{Reichhardt10} through the help of simulations, observed that the sum of the vortex velocities remains constant with increasing magnetic field. Similarly to a granular hopper, decreasing the width of the hopper aperture decreases the flow of grains. They fount that as the number of vortices increases, the pinned vortex structure becomes larger and harder to deform. This then keeps individuals from passing through the bottleneck. 

Fluxon pumps can be used to extract useful work from a fluctuating environment. The vortex ratchet can be used as a fluxon rectifier. These could be used as fluxon lenses to concentrate or disperse magnetic fields. These would have various uses including dispersing trapped flux in SQUID magnetometers. But to get the desired effects, the frequency must hit the appropriate resonance region~\cite{Wambaugh99}. This means not only the right frequency for the geometry, but also the correct temperature (i.e. random motion). They again back up the idea that the more fluxons one has, the more their motion will be restricted. 

Vortices travelling through constricting lattices can also be used to study the dynamics of interacting particles travelling through confining potentials. Yu et. al. found that with the correct geometry, vortices would begin moving as the external magnetic field varied~\cite{Yu10}. They found strong matching effects between the vortex distribution and the constriction lattice. By tailoring their diamond shaped channel, they could pick their confining potential.
 


