\chapter{Introduction}		% chapter 1
\label{introchap}

\label{section}
\section{Condensed Matter GPU simulations}
The parallel simulation of physical systems is the central pillar of this dissertation. From Clifford simulations, to cryptography, to Lennard-Jones gasses were studied during this time. This dissertation will focus on the two most promising systems, electron hop studies in thermoelectrics, and vortex jamming/pinning in superconductors. 

\section{Thermoelectrics}
Thermoelectric devices can be used to turn electric power into cooling or heating (Peltier effect). They can also be used to turn thermal gradients into electricity (Seebeck effect). Most modern refrigeration techniques require compression of some fluid which by defintion means moving parts. A Peltier device can do this much more simply. The way they work is similar to sweat on human skin. In a random assortment of water molecules on a surface, typically the "hot" ones will leave, thereby cooling that surface. If one adds a breeze, this process can occur much more efficiently. In a nutshell, this is what is happening in a Peltier device. Electrons are attempting to move heat from one place to another, but the whole process is artificially skeewed when a voltage is applied. Because this is an electron-based effect, we want materials with a high electric conductivity. On the other hand, if the thermal conductivity is too high, then the thermal gradient will suffer and heat engine efficiency will be diminished. Currently Peltier devices can only acheive 12\% of maximum theoretical efficiency compared to compressor refrigerators which can acheive 60\%. By constructing artificial nanosolids, we can manipulate which electrons can transfer heat, thereby dictating the thermal conductivity ~\cite{glatz09}. But for this to work, we need to figure out the correct specifications of these nanosolids. We could spend years building them and experimentally testing them, but with the underlying physics being well-understood it is easier to simulate them. For this, we wrote {\sc dias}. To differentiate ourselves from other Mott simulators, we did not limit ourselves to nearest neighbors.  By using Nvidia's {\sc cuda}, we could parallelize the code on GPUs which allowed all cell probabilities to be measured at the same time with no cost to performance. We used GTX-570 video cards to run these simulations. There are a lot of regimes to study in variable-range hopping.

\section{Vortex pinning}
Superconductors are valuable for an ever-expanding range of uses. Some prominent examples are magnets, qubits, voltage to frequency converters. The department of energy is considering the posibility of transfering large amounts of current losslessly across large distances. Theoretically this should be possible with superconductors. Practically we come across the problematic phenomena known as vortices.  In this dissertation we study this effect on the induced voltage in the system. A "hand-waving" approach to this helps us understand why vortices are important. They can be seen as a magnetic flux in the direction of the field. If a current is then applied perpendicular to this field, the vortices will move in a direction perpendicular to these two vectors. This movement will due to the Lorentz force, which will drain energy from the system. This drain is then measured from the voltage gap which appears in the material. Dr. Glatz wrote a program called {\sc GL} which models the important parameter function $\psi$, and displays all relevant observables. This program models a discretized system which is initialized with a parameter function and then each timestep is defined by following the Ginzburg-Landau equation. Using this program, we studied many situations such as grids of non-superconducting inclusions, funnels, vortex ratchets, and critical currents. 


