\chapter{Introduction}		% chapter 1
\label{introchap}

\label{section}

\section{Geometric Pinning in Superconductors}

The resistance of most materials drops when cooled. Some of these materials even become superconductive. That means electrons can no longer be considered particles bouncing around from ion to ion. Instead, if the material conditions are right, they form a Cooper pair condensate which then can propagate losslessly throughout the system. A Cooper pair is a boson composed of bound pairs of electrons. Phenomenologically they must be thought of as a wave-function ~\cite{Miszczak15}. Currents in an ideal superconductor, once set up, will continue to propagate indefinitely. Setting up a current in a superconductor is as simple as placing a magnet near one. The magnet will cause electrons to propagate in said superconductor such that the external magnetic field is cancelled out. This kicking out of magnetic fields from superconductors is called the Meissner effect and is the second fundamental property of a superconductor.

 A superconductor is different than a perfect conductor however. Following Maxwell's laws one would expect a perfect conductor to absorb and keep any magnetic field that one places near it. This is not the case for superconductors. They will try to kick out any magnetic field that is placed nearby, as a perfect diamagnet would do. This is also the starting point to differentiate type 1 superconductors from type 2. Both type superconductors have a saturation magnetic field. After the magnetic field reaches a certain point, the superconducive properties of the material are removed. In type 2 superconductors, Vortices push their way into the superconductor in sufficiently high magnetic fields and can be seen as single quanta of magnetic flux made up of supercurrents circulating around normal cores ~\cite{Kwok16}. The reason being that the wavefunction which governs the field must be coherent and so the phase must go through a factor of $2\pi$ in order to line up with itself around the core. In the core, the superconductive order parameter is suppressed.  
Superconductors are valuable for an ever-expanding range of uses. Some prominent examples are magnets, qubits, voltage to frequency converters. It would be greatly beneficial if energy could be transfered losslessly across large distances. Theoretically this should be possible with superconductors. Practically we come across the problematic phenomena known as dissipation.  In this dissertation we study this effect on the induced voltage in the system. A "hand-waving" approach to this helps us understand why vortices are important. They can be seen as a magnetic flux in the direction of the field. If a current is then applied perpendicular to this field, the vortices will move in a direction perpendicular to these two vectors. This movement, due to the Lorentz force, will dissipate energy from the system. This dissipation is then measured from the voltage gap which appears in the material. In the first part of this dissertation we study the dissipative effects in various situations. There we use large-scale numerical simulations of the time demendent Ginzburg Landau equation.


\section{Dynamics In Artificial Solids}
In the second part of this dissertation, we worked on a novel refrigeration strategy. In 1755, the first thermodynamic refrigerator was created by William Cullen. He used a pump to lower the pressure over diethyl ether, which caused it to evaporate and absorb heat~\cite{Chandra??}. He learned that using phase transitions were one way to transport heat. A liquid sticks together because these particles are attracted to each other and have "fallen together" into a lower energy state. It then makes sense that in order to break these particles apart (boiling), latent heat must be supplied. In Cullen's experiment this energy came from the thermal energy of the surroundings, thereby cooling the system. One can also see this from the point of view of entropy. Entropy is a quantification of the randomness of a system. In other words, how many possible states a system could have been in. The second law of thermodynamics states that entropy can only stay the same or increase. In this case, when the liquid expanded to a gas, the number of possible states of the system dramatically increased as there was more volume for the molecules to occupy. This then led to a cooling of the system. 
	Mathematically, entropy has two definitions. The first relates heat energy being exchanged at different temperatures. That is, the entropy of the system can be changed by moving around the same amount of heat at different temperatures. The second relates the number of possible states to the entropy of the system via Boltzmann's constant. In other words, as the number of possible states goes up, so does the entropy.
	Since then, engineers have found better, more efficient ways of creating temperature differentials. Better liquids which had more sought after phase characteristics such as freon were discovered. The system was also transformed into a cycle which could be used to pump heat out of the inside of the refrigerator and pump it to the outside. Any material that has the ability to change it's entropy can be used to create a thermal gradient. One of the more exotic examples is the rubber band refrigerator. When a rubber band is stretched quickly, it increases in temperature. This is because rubber bands are long chains of randomly wound up polymers. When these polymers are pulled appart and stretched, the number of posible states decreases as they are forced into a straight line. The entropy has to go somewhere which ends up being the thermal noise (i.e. temperature) of the system. This heat should then be released into the environment. Once cooled to ambient temperature, the rubber band can then be placed in whatever enclosure one is trying to cool. There, the rubber band can be allowed to contract again. The chains now seeing that they have a lot more possible states of entropy, cool down and absorb heat from the system. The process can then be started over again by taking the rubber band out of the enclosure and stretching it out ~\cite{Brown63}. This process is called a heat cycle and is the basis of all refrigerators. Even without a phase, one can move heat around by changing the system's pressure. This can be in the form of a gas, as in the case of stirling engines, or in the form of electrons as in the case of Seebeck devices. 
	In 1821 Thomas Seebeck discovered that when two metals of different temperatures were put together, a compass needle would be slightly deflected. This is because the heat caused electron energy levels to shifted differently in each metal. This led to a current when the metals were placed together, which caused a magnetic field which deflected the compass~\cite{Dommelen13}. Thermoelectric devices can be used to turn thermal gradients into electricity (Seebeck effect). They can also be used to turn electric power into cooling or heating (Peltier effect). Most modern refrigeration techniques require compression of some fluid which by defintion means mechanical parts. They also involve some sort of fluid that can degrade, corrode, or escape. A Peltier device can do this in principle without these disadvantages. A qualitative explanation to how they work is as follows; electrons in the device can be pushed towards one side using an electric field. That side will become hot due to the thermodynamic "pressure". Conversely, the side losing electrons becomes cold. This process is reversed in systems where the charge carrier is positive.

	Once again, technology found better materials which could be used to cool the system better. The next generation of thermal gradient to electric current conversion will be based on careful alteration of the thermoelectric properties of materials ~\cite{Sparks16}. The point of the alterations is to create a material which is insulating enough to keep the temperature gradient from equilibrating, yet conductive enough to let electrons through. The Wiedemann-Franz law tells us that materials which are good electrical conductors will also tend to be good thermal conductors. We also want a good Seebeck coefficient, i.e. a small thermal gradient to have a strong effect on currents. Novel nanosolid materials can achieve this feat and push the boundaries of their thermoelectric capability. Currently Peltier devices can only acheive 12\% of maximum theoretical efficiency compared to compressor refrigerators which can acheive 60\%. By constructing artificial nanosolids, we can manipulate which electrons can transfer heat, thereby dictating the thermal conductivity. Nanosolids also have the ability to scatter phonons, thereby further insulating the system thermally ~\cite{glatz09}. By tuning the properties of these granules, the figure of merit can be increased. One way to tune these is by using mixes of different granules, otherwise known as multiphasic nano-solids. Precise tuning would take many iterations of experimentation~\cite{Sparks16}. Our approach was to write a parallel Monte-Carlo simulator called {\sc dias} (Dynamics In Artificial nanoSolids). With this, we model the electrons' variable range hopping properties. Because electrons conduct most of the heat, and all of the electricity, We can then know the system's thermoelectric properties. Thus by creating a thermal gradient and comparing it to an electric gradient, we find the system's Seebeck coefficient. 

The nano-grain electron system is not entirely understood analytically. However, it shares enough of its properties with the more complete Coulomb glass model that it makes sense to study the Coulomb glass as well. The Coulomb glass is a microscopic system where electrons attempt to transport by tunneling, yet get in each other's way due to the Coulomb blockade. 

\section{Condensed Matter GPU simulations}
 We could spend years building them and experimentally testing artificial nanosolids, but with the underlying physics being well-understood it is easier to simulate them.  We could even spend years running these simulations in series on a CPU. The parallel simulation of physical systems is the central pillar of this dissertation.  By using Nvidia's {\sc cuda} and GTX-570 video cards, we parallelize the code on GPUs which allow all cell probabilities to be measured at the same time with no cost to performance. This dissertation will focus on the two most promising systems;vortex jamming/pinning in superconductors, and electron hop studies in thermoelectrics. We begin by explaining some of the theory and background of the GPU simulation of the time dependent Ginzburg-Landau system. Then we show how jamming and inclusion-vortex commensurability increases the depinning current. Afterwards, we then go into some depth on Dias and how we can use it to predict Seebeck coefficients. 

