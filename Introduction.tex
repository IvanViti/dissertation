\chapter{Introduction}		% chapter 1
\label{introchap}

\label{section}

\section{Thermoelectrics}
	In 1755, the first thermodynamic refrigerator was created by William Cullen. He used a pump to lower the pressure over diethyl ether, which caused it to evaporate and absorb heat~\cite{Chandra??}. The phases of matter are caused by inter-molecular forces. A liquid sticks together because these particles are attracted to each other and have "fallen together" into a lower energy state. It then makes sense that in order to break these particles apart (boiling), energy must be supplied. In Cullen's experiment this energy came from the thermal energy of the surroundings, thereby cooling the system. One can also see this from the point of view of entropy. Entropy is a quantification of the randomness of a system. In other words, how many possible states a system could have been in. Entropy can only stay the same or increase. In this case, when the liquid expanded to a gas, the number of possible states of the system dramatically increased as there was more volume for the molecules to occupy. This then led to a cooling of the system. 
	Mathematically, entropy has two definitions. The first:
\begin{eqnarray}
dS = \frac{dQ}{T},
\label{dS}
\end{eqnarray}
where $dS$ is the change in entropy, $dQ$ is the change in heat, and $T$ is the temperature at which this exchange is occuring. Right away, we can see that the entropy of the system can be changed by moving around the same amount of heat at different temperatures. The second is a bit more esoteric and is:
\begin{eqnarray}
S = K_b ln (W),
\label{S}
\end{eqnarray}
where $K_b = 1.38064852 \times 10^{23}$ is the Boltzmann constant, and W is a measure of the number of states. In other words, as the number of possible states goes up, so does the entropy.

	Since then, engineers have found better, more efficient ways of creating temperature differentials. Better liquids which had more sought after phase characteristics such as freon were discovered. The system was also transformed into a cycle which could be used to pump heat out of the inside of the refrigerator and pump it to the outside. Any material that has the ability to change it's entropy can be used to create a thermal gradient. One of the more exotic -yet simple- examples is the rubber band refrigerator. When a rubber band is stretched quickly, it increases in temperature. This is because rubber bands are long chains of randomly wound up polymers. When these polymers are pulled appart and stretched, the number of posible states decreases as they are forced into a straight line. The entropy has to go somewhere which ends up being the thermal noise (i.e. temperature) of the system. This heat should then be released into the environment. Once cooled to ambient temperature, the rubber band can then be placed in whatever enclosure one is trying to cool. There, the rubber band can be allowed to contract again. The particles now seeing that they have a lot more possible states of entropy, cool down and absorb heat from the system. The process can then be started over again by taking the rubber band out of the enclosure and stretching it out ~\cite{Brown63}. This process is called a heat cycle and is the basis of most refrigerators. 
        The next generation of thermal gradient to electric current conversion will be based on thermoelectric materials ~\cite{Sparks16}. Novel nanosolid materials can push the boundaries of their thermoelectric capabilitity. These materials are artificially created by joining nano-scale granules together at comparable distances. The number that denotes usefulness of a thermoelectric device is the dimensionless figure of merit
\begin{eqnarray}
ZT = \frac{\sigma S^2 T}{k},
\label{ZT}
\end{eqnarray}
where $\sigma$ is the conductivity, $S^2$ is the Seebeck coefficient, $T$ is the temperature, and $k$ is the thermal conductivity ~\cite{chen}. The Seebeck coefficient is a relationship between the electric gradient and the thermal gradient:
\begin{eqnarray}
S = \frac{V}{\Delta T},
\label{Seebeck}
\end{eqnarray}
 By tuning the properties of these granules, the figure of merit can be increased. Precise tuning would take many iterations of experimentation. Our approach was to write a parallel Monte-Carlo simulator. With this, we model the electrons' variable range hopping properties. Because electrons conduct most of the heat, and all of the electricity, We can then know the system's thermoelectric properties. Thus by creating a thermal gradient and comparing it to an electric gradient, we find the system's figure of merit. The ultimate goal being to be able to transport electricity while keeping our thermal gradient as steep as possible. 

Thermoelectric devices can be used to turn electric power into cooling or heating (Peltier effect). They can also be used to turn thermal gradients into electricity (Seebeck effect). Most modern refrigeration techniques require compression of some fluid which by defintion means moving parts. They also involve some sort of fluid that can degrade, corrode, or escape. A Peltier device can do this much more simply. The way they work is similar to sweat on human skin. In a random assortment of water molecules on a surface, typically the "hot" ones will leave, thereby cooling that surface. If one adds a breeze, this process can occur much more efficiently. In a nutshell, this is what is happening in a Peltier device. Electrons are attempting to move heat from one place to another, but the whole process is artificially skeewed when a voltage is applied. Because this is an electron-based effect, we want materials with a high electric conductivity. On the other hand, if the thermal conductivity is too high, then the thermal gradient will suffer and heat engine efficiency will be diminished. Currently Peltier devices can only acheive 12\% of maximum theoretical efficiency compared to compressor refrigerators which can acheive 60\%. By constructing artificial nanosolids, we can manipulate which electrons can transfer heat, thereby dictating the thermal conductivity ~\cite{glatz09}. But for this to work, we need to figure out the correct specifications of these nanosolids. We could spend years building them and experimentally testing them, but with the underlying physics being well-understood it is easier to simulate them. For this, we wrote {\sc dias}. To differentiate ourselves from other Mott simulators, we did not limit ourselves to nearest neighbors.  By using Nvidia's {\sc cuda}, we could parallelize the code on GPUs which allowed all cell probabilities to be measured at the same time with no cost to performance. We used GTX-570 video cards to run these simulations. There are a lot of regimes to study in variable-range hopping.

\section{Vortex pinning}

As most materials are cooled, their resistance decreases. At these temperatures, electrons can no longer be considered particles bouncing around from ion to ion. Instead, they must be thought of as quantum-mechanical waves ~\cite{Miszczak15}. While this is inherently a quantum-mechanical phenomenon, it has a myriad of macroscopic effects. Currents in an ideal superconductor, once set up, will continue to propagate indefinitely. Setting up a current in a superconductor is as simple as placing a magnet near one. The magnet will cause electrons to propagate in said magnet until it's magnetic field is cancelled out. This kicking out of magnetic fields from superconductors is called the Meissner effect. This effect is a consequence of London's equations. We begin with Ampere's law:
\begin{eqnarray}
\nabla \times \overrightarrow B  = \mu_0 \overrightarrow J,
\label{Ampere}
\end{eqnarray}
where $\overrightarrow B$ is the magnetic field, $\mu_0$ is the permeability of free space, and $\overrightarrow J$ is the current. We then use the vector identity 

\begin{eqnarray}
\nabla \times \nabla \times \overrightarrow A  = -\nabla^2 \overrightarrow A,
\label{stokes}
\end{eqnarray}
to get 
\begin{eqnarray}
\nabla^2 B = \frac{B}{lambda^2},
\label{penetration}
\end{eqnarray}

where $\lambda$ is the penetration depth. 

 A superconductor is different than a perfect conductor however. Following Maxwell's laws one would expect a perfect conductor to absorb and keep any magnetic field that one places near it. This is not the case for superconductors. They will try to kick out any magnetic field that is placed nearby. This is also the starting point to differentiate type 1 superconductors from type 2. Type 1 superconductors have a saturation magnetic field. After the magnetic field reaches a certain point, the superconducive properties of the material are removed. In type 2 superconductors, the field is instead let through in small islands by the material. These small islands tend to have a circular shape as it is the configuration which allows the most magnetic field to pass while taking up the least energy. That is because the amount of energy required for this is a function of the inter-phase surface area. The amount of flux that can sneak in tends to be quantized. The reason being that the wavefunction which governs the field must be continuous and so it must go through a factor of $2\pi$. This then gives us a quantization of field allowed in each flux line.  
Superconductors are valuable for an ever-expanding range of uses. Some prominent examples are magnets, qubits, voltage to frequency converters. The department of energy is considering the posibility of transfering large amounts of current losslessly across large distances. Theoretically this should be possible with superconductors. Practically we come across the problematic phenomena known as vortices.  In this dissertation we study this effect on the induced voltage in the system. A "hand-waving" approach to this helps us understand why vortices are important. They can be seen as a magnetic flux in the direction of the field. If a current is then applied perpendicular to this field, the vortices will move in a direction perpendicular to these two vectors. This movement will due to the Lorentz force, which will drain energy from the system. This drain is then measured from the voltage gap which appears in the material. Dr. Glatz wrote a program called {\sc GL} which models the important parameter function $\psi$, and displays all relevant observables. This program models a discretized system which is initialized with a parameter function and then each timestep is defined by following the Ginzburg-Landau equation. Using this program, we studied many situations such as grids of non-superconducting inclusions, funnels, vortex ratchets, and critical currents. 

\section{Condensed Matter GPU simulations}
The parallel simulation of physical systems is the central pillar of this dissertation. From Clifford simulations, to cryptography, to Lennard-Jones gasses were studied during this time. This dissertation will focus on the two most promising systems, electron hop studies in thermoelectrics, and vortex jamming/pinning in superconductors. We begin by explaining some of the theory and background of these two systems. We then go into some depth on Dias and how it works. Finally we present results from Dias and GL. 

